\documentclass[10pt]{article}
\usepackage[margin=0.5in]{geometry}
\usepackage{amsmath, amsthm, amssymb}
\usepackage{mathtools}
\usepackage{hyperref}
\usepackage{url}

\usepackage{titlesec}
\titlelabel{\thetitle.\quad}

\DeclareMathOperator*{\argmax}{arg\,max}
\DeclareMathOperator*{\argmin}{arg\,min}

\newtheorem{definition}{Definition}
\newtheorem{remark}{Remark}
%\let\oldref\ref
%\renewcommand{\ref}[1]{((\oldref{#1}))}

\newcommand\sbullet[1][.5]{\mathbin{\vcenter{\hbox{\scalebox{#1}{$\bullet$}}}}}
\newcommand\numberthis{\addtocounter{equation}{1}\tag{\theequation}}
\newcommand{\bigzero}{\mbox{\normalfont\Large\bfseries 0}}
\newcommand{\rvline}{\hspace*{-\arraycolsep}\vline\hspace*{-\arraycolsep}}

\title{\vspace{-2.0em} \vspace{-0.5em}}
\author{Matt Piekenbrock}
\date{}

\begin{document}
\noindent

\section{Motivation}
$<$ insert motivating examples, etc $>$

% Suppose one observes points in a geometric space whose position is driven by some unknown continuous-time system. 
% Towards understanding its dynamic, one may ask whether one can infer properties of the underlying evolving system
\section{Background \& Notation}
% \textbf{Persistent Homology:}
A \emph{simplicial complex} $K \subseteq \mathcal{P}(V)$ over a vertex set $V$ is a collection of simplices $\{\sigma\}$ such that $\tau \subseteq \sigma \in K \implies \tau \in K$. A $p$-chain is a formal $\mathbb{F}$-linear combination of $p$-simplices of $K$. The collection of $p$-chains under addition yields an $\mathbb{F}$-vector space denoted $C_p(K)$. 
The $p$-boundary $\partial_p(\sigma)$ of a $p$-simplex $\sigma\in K$ is the alternating sum of its oriented co-dimension 1 faces, and the $p$-boundary of a $p$-chain is defined linearly in terms of its constitutive simplices. 
A $p$-chain with zero boundary is called a $p$-cycle, and together they form $Z_p(K) = \mathrm{Ker}\,\partial_p$. Similarly, the collection of $p$-boundaries forms  $B_p(K) = \mathrm{Im}\,\partial_{p+1}$. Since $\partial_p \circ \partial_{p+1} = 0$ for all $p\geq 0$, the quotient space $H_p(K) = Z_p(K) / B_{p}(K)$ is well-defined, and $H_p(K)$ is called the $p$-th homology of $K$ with coefficients in $\mathbb{F}$. 

A \emph{filtration} $K_\bullet$ of a simplicial complex $K$ is a function $f : K \to \mathbb{R}$ satisfying $f(\tau) \leq f(\sigma)$ whenever $\tau \subseteq \sigma$. Connecting a sequence of simplices $[\sigma_i]_{i=1, \dots, m}$ ordered increasingly by $f$ by inclusion yields such a family of complexes:
\begin{equation}
	\emptyset = K_0 \subsetneq K_1 \subsetneq \dots \subsetneq K_m  = K_\bullet
\end{equation} 
where $K_i  = K_{i-1} \cup \{\sigma_i\}$. For every pair $i,j \in [m]$ with $i < j$, the inclusions $K_i \subsetneq K_{i+1} \subsetneq \dots \subsetneq K_j$ induce linear transformations $h_p^{i,j}$  at the level of homology:
\begin{equation}
	0 = H_p(K_0) \to \dots \to H_p(K_i) \underbracket[0.5pt]{\to \dots \to}_{h_p^{i,j}} H_p(K_j) \to \dots \to H_p(K_m) = H_p(K_\bullet) 
\end{equation}
The $p$-th persistent homology groups are the images of these transformations: $H_{p}^{i,j} = \mathrm{Im}\,h_p^{i,j}$. 
When persistent homology is considered with field coefficients $\mathbb{F}$, the sequence of homology groups admits a unique decomposition of $K_\bullet$ into a pairing of simplices $(\sigma_i, \sigma_j)$~\cite{}. These pairs demarcate the evolution of homology classes: $\sigma_i$ marks the creation of a homology class, $\sigma_j$ marks its destruction, and the difference $\lvert i - j \rvert$ records the lifetime of the class, called its \emph{persistence}. For a fixed $p \geq 0$, the collection of persistent pairs $(f(\sigma_i), f(\sigma_j))$ together with unpaired simplices $(f(\sigma_l), \infty)$ form a summary representation $\mathrm{dgm}_p(K_\bullet)$ called the \emph{$p$-th persistence diagram of $K_\bullet$}.

\begin{remark}
\normalfont In practice, filtrations often arise from triangulations parameterized by geometric scaling parameters, and references to the ``persistence'' of a given homology class actually refers to its lifetime $\lvert f(\sigma_i) - f(\sigma_j) \rvert$ with respect to the scaling parameter.
$<$TODO: introduce Rips, switch notation accordingly $>$

For a fixed $t \in \mathrm{T}$, the \emph{Vietoris-Rips} complex at scale $\epsilon \in \mathbb{R}$ is the abstract simplicial complex given by 
$$\mathrm{Rips_{\epsilon}}(d_X(t)) := \{ \sigma \subset X : d_X(t)(x, x') \leq \epsilon \text{ for all } x, x' \in \sigma \} $$ 
\noindent Connecting successive complexes via inclusion maps $\mathrm{Rips_{\epsilon}}(d_X(t)) \hookrightarrow \mathrm{Rips_{\epsilon'}}(d_X(t))$ for $\epsilon < \epsilon'$ yields a family of complexes $\mathrm{Rips}_{\alpha} := \{ \, \mathrm{Rips}_\epsilon(d_X(t)) \, \}_{\epsilon \leq \alpha}$ which we call the \emph{Vietoris-Rips filtration at time $t$}. 
These inclusions induce linear maps at level of homology, i.e. 
$$ \mathrm{H}_p(\mathrm{Rips}_{\epsilon}(d_X(t))) \to \mathrm{H}_p(\mathrm{Rips}_{\epsilon'}(d_X(t))) \to \dots \to \mathrm{H}_p(\mathrm{Rips}_{\alpha}(d_X(t)))$$ 
where $0 \leq \epsilon \leq \epsilon' \leq \alpha$. 

\end{remark} 

%Note that if $i = j$, then $H_{p}^{i,j} = H_{p}(K_i) = H_{p}(K_i)$ is   just the ``standard'' homology. 
% Simplices whose inclusion in the filtration creates a new homology class are called \emph{creators}, and simplices that destroy homology classes are   called \emph{destroyers}. 
% The filtration indices of these creators/destroyers are referred to as \emph{birth} and \emph{death} times, respectively. 
%The collection of birth/death pairs $(i,j)$ is denoted $\mathrm{dgm}_p(K)$, and referred to as the $p$-th \emph{persistence diagram} of $K$.
%If a homology class is born at time $i$ and dies entering time $j$, the difference $\lvert i - j \rvert$ is called the \emph{persistence} of that class.



\section{Methodology}
The goal of this section is to understand the main methodology of the paper.

% Suppose one is given a time-varying metric space $(X, d_X(\cdot))$ and is interested in finding an instance in time where the persistence diagram contains many highly persistent pairs.   

\subsubsection*{Persistent Betti Numbers:} 
Let $B_p(K_\ast) \subseteq Z_p(K_\ast) \subseteq C_p(K_\ast)$ denote the $p$-th boundary, cycle, and chain groups of $K_\ast$, respectively. 
Given a simplicial filtration $K_{\bullet}$, let boundary operator $\partial_p : C_p( K_{\bullet}) \to C_p(K_{\bullet})$ denote the boundary operator sending $p$-chains to their respective boundaries. 
With a slight abuse of notation, we also use $\partial_p$ to also denote the filtration boundary matrix with respect to the ordered basis $(\sigma_i)_{1 \leq i \leq m_p}$.  
The $p$-th persistent Betti number between scales $(b,d)$ is defined as: 
\begin{align*}
	\beta_p^{b,d} &= \mathrm{dim}(H_p^{b,d}) \\
	&= \mathrm{dim} \left( Z_p(K_b) / (Z_p(K_b) \cap B_p(K_d) \right) \\
	& \numberthis = \mathrm{dim} \left( Z_p(K_b) \right) - \mathrm{dim}\left( Z_p(K_b) \cap B_p(K_d) \right ) \label{eq:pb2}
\end{align*}
Note that, since $\mathrm{dim}(C_p(K_\ast)) = \mathrm{dim}(B_{p-1}(K_\ast)) + \mathrm{dim}(Z_p(K_\ast))$, we may use the rank-nullity theorem torewrite~\eqref{eq:pb2} as:
\begin{equation} \label{eq:pb3}
\beta_p^{b,d} = \mathrm{dim} \left( C_p(K_b) \right) - \mathrm{dim} \left( B_{p-1}(K_b) \right) - \mathrm{dim}\left( Z_p(K_b) \cap B_p(K_d) \right )  
\end{equation}
Let $\partial_p^{b}$ and $\partial_p^{b, d}$ denote matrices whose columns span the subspaces $B_{p-1}(K_b)$ and $Z_p(K_b) \cap B_p(K_d)$, respectively. We address their computation in section~\eqref{sec:computation}. Substituting these matrices appropriately, equation~\eqref{eq:pb3} can be written as: 
\begin{equation}\label{eq:pb_rank}
	\beta_p^{b,d} = \lvert \, \partial_p^b \, \rvert - \mathrm{rank}(\partial_p^b) - \mathrm{rank}(\partial_p^{b,d}) 
\end{equation}
where $\lvert \, M \, \rvert = \mathrm{dim}(\mathrm{dom}(M))$. Thus, $\beta_p^{b,d}$ is expressible as a difference between a simple-to-compute quantity ($\lvert \, \partial_p^b \, \rvert$ simply counts the number of $p$-simplices with filtration value $f(\sigma) \leq b$ for some fixed $b \in \mathbb{R}_+$) and the rank of two particular matrices. We make use of this fact in later sections of the paper. 

\subsubsection*{Time-varying setting:}
Here, we formally express our objective (equation~\eqref{eq:pb_rank}) in the time-varying setting. 
Let $\delta_X$ denote an $\mathrm{T}$-parameterized metric space $\delta_X = ( X, d_X(\cdot) )$, where $d_X: \mathrm{T} \times X \times X \to \mathbb{R}_+$ is called a \emph{time-varying metric}  and $X$ is a finite set with fixed cardinality $\lvert X \rvert = n$. $\delta_X$ as called a \emph{dynamic metric space} (DMS) iff $d_X(\cdot)(x, x'): \mathbb{R} \to \mathbb{R}_{+}$ is continuous for every pair $x, x' \in X$ and $(X, d_X(t))$ is a pseudo-metric space for every $t \in \mathrm{T}$. 

The family of Betti numbers varying continuously with time has been studied before in~\cite{}, where they are used to discern topological features persistent in both time and space. In this setting, the 

% The $p$th \emph{Betti number} is defined as the dimension of any of these homology groups $\beta_p = \mathrm{dim}(H_p(\mathrm{Rips}_{\alpha})))$. 
By restricting our attention to the persistent homology groups which were born before $b \in \mathbb{R}$ and died after $d \in \mathbb{R}$, we obtain the $p$-th \emph{persistent Betti number} with respect to $(b,d)$ as a function of $t \in \mathrm{T}$: 
$$ \beta_{p}^{b,d}(t) = \left(\mathrm{dim} \circ \mathrm{H}_p^{b,d} \circ \mathrm{Rips}_d \circ d_X \right)(t)$$
This quantity can be readily visualized as the number of persistent pairs lying inside the box $[0, b] \times (d, \infty)$  on the collection of all persistence diagrams for varying $t \in \mathrm{T}$.
We consider the problem of maximizing the $p$-th \emph{persistent} Betti number $\beta^{b,d}_p$ over $\mathrm{T}$: 
\begin{equation}
	t_\ast = \argmax_{t \in \mathrm{T}}	 \beta_{p}^{b,d}(t)
\end{equation}
Since Betti numbers are integer-valued invariants, direct optimization is difficult. Moreover, the space of persistence diagrams is [banach space statement]....
Nonetheless, the differentiability of persistence has been studied extensively in [show chain rule paper on persistence diagrams]...


% For the moment, we omit the subscript $t \in \mathrm{T}$ and focus on a fixed instance of time. 



\subsubsection*{Continuous Relaxation TODO}
As an integer-valued invariant, Betti numbers pose several difficulties to direct optimization. Thus, we require alternative expressions for each of the terms in equation~\eqref{eq:pb_rank} to extend its applicability to the time-varying setting.

We begin by extending the standard definition of an elementary $p$-chain to the dynamic setting. Recall a $p$-chain of a simplicial filtration $K_\bullet$ with coefficients in $\mathbb{F}$ is a function $c_p$ on the oriented $p$-simplices of $K$ satisfying $c_p(\sigma) = -c_p(\sigma')$ if $\sigma$ and $\sigma'$ are opposite orientations of the same simplex, and $c_p(\sigma) = 0$ otherwise. 
A $p$-chain is called \emph{elementary with respect to $q \in \mathbb{F}$} if it satisfies:
\begin{align*}
	c_p(\sigma) &= +q  \quad & \\
	c_p(\sigma') &= -q \quad &\text{if } \sigma' \text{ is the opposite orientation of }\sigma \\
	c_p(\tau) &= 0 \quad & \text{otherwise}
\end{align*}
Once all $p$-simplices of $K$ are oriented, each $p$-chain can be written unique as a finite linear combination $c_p = \sum_{i=0}^p n_i \sigma_i$ 
of the corresponding elementary chains $\sigma_i$. 
\begin{equation}
	\partial_p(\sigma_i) = \partial_p[ v_0, \dots, v_p ] = \sum\limits_{i = 0}^p q(-1)^i [v_0, \dots, \hat{v_i}, \dots, v_p]
\end{equation}
where the notation $\hat{v}_p$ means that $v_p$ is excluded in the $i$-th summand, and $[v_0, \dots, v_p]$ denotes the oriented simplex. 
%\begin{definition}[Time-varying elementary $p$-chain]
%	An elementary $p$-chain $c_p : T \times K$ is said to be time-varying if $c_p(\cdot)(\sigma) = f(\sigma; t)$ is continuous in $T$. 
%\end{definition}
\begin{definition}[Time-varying boundary matrix]
A time-varying boundary matrix is a matrix-realization of the operator $\partial_p : C_p(K_\bullet) \times T \to C_p(K_\bullet(t))$ with respect to a time-varying total order $f(\sigma)$. 
\end{definition}

We begin by fixing $\mathbb{F} = \mathbb{R}$ as our choice of coefficients. At the algebraic level, persistent homology admits a canonical decomposition for coefficients in any choice of field~\cite{}, though at the expense of torsion information\footnote{}.  

To make the entries of  $\partial_p^\ast$ vary continuously in $t \in T$, we replace the entries of $\partial_p^\epsilon$ with time-dependent quantity $c_p(\sigma) = \epsilon - \mathrm{diam}(\sigma)$. 
Note that $\mathrm{diam}_t(\sigma) = \{ \max(d_X(t)(x, x')) : x,x' \in \sigma\}$ must vary continuously in $t \in T$ by the definition of DMS.
%At the algorithmic level, the choice of field coefficient affects the practical implementation 

For a fixed $t \in T$, we obtain a boundary matrix $\partial_{p}^{b,d}$ up to filtration value (diameter) $d \in \mathbb{R}$ for $d_X(t)$. We recall the integer-valued function (equation~\eqref{eq:pb_rank}) we would like to relax. To do this, we substitute the nuclear norm $\lVert \, \cdot \, \rVert_\ast$  for the $\mathrm{rank}$ function and a sigmoid-like function $S_b : K \to \mathbb{R}_{+}$ for the order function $\lvert \, \cdot \, \rvert$, obtaining: 
\begin{equation}\label{eq:relaxation_pb}
\hat{\beta}_p^{b,d} = S_b(K) - \lVert \partial_p^b \rVert_{\ast} - \lVert \partial_p^{b,d} \rVert_\ast
\end{equation} 
where $S_b(K) = \sum_{\sigma \in K} \mathrm{sigmoid}(\lvert b - \mathrm{diam}(\sigma)\rvert)$.
Our choice of the nuclear norm is motivated by the fact that it is often used due to its close relationship to the rank function, as first observed by Fazel et al~\cite{} (we discuss this more in section~\ref{}). 

%$<$ TODO: the goal $>$
%First, we that prove the following properties of equation~\eqref{eq:relaxation_pb}:
%\begin{enumerate}
%	\item If $t^\ast = \argmin\limits_{t \in T} \beta_{p}^{b,d}$ and $\hat{t}^\ast = \argmin\limits_{t \in T} \hat{\beta}_{p}^{b,d}$, then $t^\ast = \hat{t}^\ast$
%	\item $\hat{\beta}_{p}^{b,d}(t)$ is continuous as a function of $t \in T$
%	\item $\hat{\beta}_{p}^{b,d}(t)$ admits a subgradient $\hat{\beta}_{p}^{b,d}(t)$
%\end{enumerate}
%We first begin with properties (2) and (3). (2) is obvious... To see (3), consider:
Equation~\eqref{eq:relaxation_pb} admits a differentiable form amenable to optimization. 
\begin{equation}
	\nabla \hat{\beta}_p^{b,d} = \nabla S_b(K) - \nabla \lVert \partial_p^b \rVert_{\ast} \cdot J_b - \nabla \lVert \partial_p^{b,d} \rVert_\ast \cdot J_{b,d}
\end{equation}
For any matrix $M \in \mathbb{R}^{n \times m}$ whose corresponding singular value decomposition (SVD) is $M = U \Sigma V^T $, the characterization of the (sub)gradient of $\lVert M \rVert_\ast$ is given by\cite{}: 
\begin{equation}
	\partial\|M\|_{*}=\left\{U V^T + W: P_{U} W=0, W P_{V}=0,\|W\| \leq 1\right\}
\end{equation}
where $P_U$ ($P_V$, resp.) is an orthogonal projector onto the column space of $U$ ($V$, resp.). For simplicity we set $W = 0$ and obtain: % TODO: write as functional way
 \begin{equation}
	\nabla \hat{\beta}_p^{b,d} = \nabla S_b(K) - U_b V_b^T J_b - U_{b,d} V_{b,d}^T J_{b,d}
\end{equation}

% Equipping the set of all simplices $\mathcal{P}(X)$ with an appropriate total order,
% Thus, after suitable normalization, the right-most term of equation~\ref{eq:block_pb} can be relaxed to a convex function. 
% The columns of $\partial_p^\ast$ span $C_p(X_\ast)$, thus 
% Let $\mathcal{N}(\cdot)$ and $\mathcal{R}(\cdot)$ denote the null-space of column-space of its arguments, respectively. 
%The $p$-th persistent Betti number is informative in capturing the necessary conditions of our goal: ...
%Ideally, we would like an expression akin to equation~\eqref{eq:block_pb} that is amenable to optimization.

\section{Computation}\label{sec:computation}
\subsection*{Bases computation}
In this section, we discuss the computation of suitable bases for the subspaces $Z_p(X_\ast)$, $B_p(K_\ast)$, and $Z_p(X_\ast) \cap B_p(X_\ast)$. In particular, we address two cases: the \emph{dense} case, wherein the aforementioned bases are represented densely in memory, and the \emph{sparse} case, which uses the structure of a particular decomposition of the boundary matrices to derive bases whose size in memory inherits the sparsity pattern of the decomposition.
\\
\\
\textbf{Sparse case:} We require an appropriate choice of bases for the groups $B_{p-1}(K_\ast)$ and $Z_p(X_\ast) \cap B_p(X_\ast)$. 
For some fixed $t \in T$, let $R_p = \partial_p V_p$ denote the decomposition discussed above, and let $b, d \in \mathbb{R}_+$ be fixed constants satisfying $b \leq d$. Since the boundary group $B_{p-1}(K_b)$ lies in the image of the $\partial_{p}$, it can be shown that a basis for the boundary group $B_{p-1}(K_\ast)$ is given by: 
\begin{flalign}
	&& M_p^b = \{ \, \mathrm{col}_{R_{p+1}}(j) \neq 0 \mid j \leq b \, \}  && span()
\end{flalign}
Moreover, since $B_{p-1}(K_b) = \mathrm{Im}(\partial_p^b)$, we have $\mathrm{span}(M_p^b) = B_{p-1}(K_b)$ and thus $\mathrm{rank}(M_p^b) = \mathrm{rank}(\partial_p^b)$. Indeed, it can be shown that every lower-left submatrix of $\partial_p^\ast$ satisfies $\mathrm{rank}(\partial_p^\ast) = \mathrm{rank}(R_p^\ast)$. Thus, although $M_p^b$ does provide a minimal basis for the boundary group $B_{p-1}(K_b)$, it is unneeded here. 

A suitable basis for the cycle group can also be read off from the reduced decomposition directly as well. Indeed, let $R_p = \partial_p V_p$ as before. Then the cycle group is spanned by linear combinations of columns of $V_p$: 
\begin{equation}
	Z_p^b = \{ \, \mathrm{col}_{V_p}(j) \mid \mathrm{col}_{R_{p}}(j) = 0, j \leq b \, \}	
\end{equation}
The formulation of a basis spanning $Z_p(K_i) \cap B_p(K_j)$ is more subtle, as we can no longer use the  fact that every lower-left submatrix of $R_p$ has the same rank as the same lower-left submatrix of $\partial_p$. 
Nonetheless, a basis for this group can be obtained by reading off specific columns from $R_p$: 
\begin{equation}
	M_p^{b, d} := \{\, \mathrm{col}_{R_{p+1}}(k) \neq 0 \mid 1 \leq k \leq d \text{ and } 1 \leq \mathrm{low}_\mathrm{R_{p+1}}(k) \leq b \, \}
\end{equation}
%\begin{flalign}
%	(\, Z_p(K_i) \cap B_p(K_j) \, ) && M_p^{b, d} := \{\, \mathrm{col}_{R_p}(k) \mid 1 \leq k \leq d \text{ and } 1 \leq \mathrm{low}_\mathrm{R_p}(k) \leq b \, \} &&
%\end{flalign}
One can show that $M_b^d$ does indeed span $Z_p(X_\ast) \cap B_p(X_\ast)$ by using the fact that the non-zero columns of $R_p$ with indices at most at most $d$ form a basis for $B_p(K_d)$, and that each low-row index for every non-zero is unique. 
%The issue here is that 
\\
\\
\noindent
\textbf{Dense case:} 
In general, persistent homology groups and its various factor groups are well-defined and computable with the reduction algorithm with coefficients chosen over any ring. By applying operations with respect to a field $\mathbb{F}$, both the various group structures $Z_p(K_\bullet) \subseteq B_p(K_\bullet)  \subseteq C_p(K_\bullet) $ and their induced quotient groups $H_p(K_\bullet)$ are vector spaces; thus, the computation of suitable bases can be approached from a purely linear algebraic perspective.
In particular, by fixing $\mathbb{F} = \mathbb{R}$, we inherit not only many useful tools for obtaining suitable bases for these groups, but also access to their corresponding optimized implementations as well. 

Consider the $p$-th boundary operator $\partial_p^\ast : C_p(K_\ast) \to C_{p-1}(K_\ast)$ whose matrix realization with respect to some choice of simplex ordering $\{\sigma_i\}_{1 \leq i \leq m}$ we also denote with $\partial_p$. By definition, the boundary group $B_p(K_\ast)$ is given by the image $\mathrm{Im}(\partial_{p+1}^\ast) = B_p(K_\ast)$, thus one may basis for $B_p(K_\ast)$ by computing the considering the first $r > 0$  columns of the reduced SVD: 
\begin{equation}
	M_p^\ast = [\, u_1 \mid u_2 \mid \dots \mid u_r \, ] = \{ \,  \, \}
\end{equation}


%$<$ TODO $>$
\section*{Optimization}

A remarkable result established by~\cite{} show that the $\mathrm{rank}(\cdot)$ function is lower-bounded by the convex envelope... [describe this more in detail]


\noindent  \textbf{DC Formulation:}
The 
\begin{align}
	\beta_p^{b,d} &= \lvert \, \partial_p^b \, \rvert - \mathrm{rank}(\partial_p^b) - \mathrm{rank}(\partial_p^{b,d}) \\
	&= \lvert \, \partial_p^b \, \rvert - \left( \mathrm{rank}(\partial_p^b) + \mathrm{rank}(\partial_p^{b,d}) \right) \\
	&=
	\lvert \, \partial_p^b \, \rvert - 
	\arraycolsep=1.8pt\def\arraystretch{1.25}
	\mathrm{rank}\left(\left[\begin{array}{c|c}
 		\partial_p^{b} & 0 \\
		\hline
		0 & \partial_p^{b,d}
	\end{array}\right] \right) \label{eq:block_pb}
\end{align}


\appendix
\section{Appendix}

\subsection*{Dynamic Metric Spaces}
Consider an $\mathbb{R}$-parameterized metric space $\delta_X = ( X, d_X(\cdot) )$ where
$X$ is a finite set and $d_X(\cdot): \mathbb{R} \times X \times X \to \mathbb{R}_{+}$, satisfying: 
\begin{enumerate}
	\item For every $t \in \mathbb{R}, \delta_X(t) = (X, d_X(t))$ is a pseudo-metric space\footnote{This is required so that if one can distinguish the two distinct points $x, x' \in X$ incase $d_X(t)(x, x') = 0$ at some $t \in \mathbb{R}$. } 
	\item For fixed $x, x' \in X$, $d_X(\cdot)(x, x'): \mathbb{R} \to \mathbb{R}_{+}$ is continuous.
\end{enumerate}
When the parameter $t \in \mathbb{R}$ is interpreted as \emph{time}, the above yields a natural characterization of a ``time-varying'' metric space. More generally, we refer to an $\mathbb{R}^h$-parameterized metric space as \emph{dynamic metric space}(DMS). Such space have been studied more in-depth~\cite{} and have been shown...
 
\subsection*{Homology}
Let $K$ be an abstract simplicial complex and $\mathbb{F}$ a field. A $p$-chain is a formal $\mathbb{F}$-linear combination  of $p$-simplices of $K$. The collection of $p$-chains under addition yields an $\mathbb{F}$-vector space denoted $C_p(K)$. 
The $p$-boundary $\partial_p(\sigma)$ of a $p$-simplex $\sigma\in K$ is the alternating sum of its oriented co-dimension 1 faces, and the $p$-boundary of a $p$-chain is defined linearly in terms of its constitutive simplices. 
A $p$-chain with zero boundary is called a $p$-cycle, and together they form $Z_p(K) = \mathrm{Ker}\,\partial_p$. Similarly, the collection of $p$-boundaries forms  $B_p(K) = \mathrm{Im}\,\partial_{p+1}$. Since $\partial_p \circ \partial_{p+1} = 0$ for all $p\geq 0$, then the quotient space $H_p(K) = Z_p(K) / B_{p}(K)$ is well-defined, and called the $p$-th homology of $K$ with coefficients in $\mathbb{F}$. 
If $\{K_i\}_{i\in [m]}$ is a filtration, then the inclusion maps  $K_i\subset K_{i+1}$ induce linear transformations 
%$f_p^{i,j}: H_p(K_i) \to H_p(K_j)$ 
at the level of homology:
\begin{equation}
	H_p(K_1) \to H_p(K_2) \to \dots \to H_p(K_m)
\end{equation}
%The $p$-th persistent homology groups are the images of these transformations: $H_{p}^{i,j} = \mathrm{Im}\,f_p^{i,j}$. 
%Note that if $i = j$, then $H_{p}^{i,j} = H_{p}(K_i) = H_{p}(K_i)$ is   just the ``standard'' homology. 
Simplices whose inclusion in the filtration creates a new homology class are   called \emph{creators}, and simplices that destroy homology classes are   called \emph{destroyers}. 
The filtration indices of 
these creators/destroyers are referred to as \emph{birth} and \emph{death} times, respectively. 
The collection of birth/death  pairs 
$(i,j)$ is denoted $\mathrm{dgm}_p(K)$, 
and referred to as the $p$-th \emph{persistence diagram} of $K$.
If a homology class is born at time $i$ and dies entering time $j$, the difference $\lvert i - j \rvert$ is called the \emph{persistence} of that class.
In practice, filtrations often arise from triangulations parameterized by geometric scaling parameters, and the ``persistence'' of a homology class actually refers to its lifetime with respect to the scaling parameter. 


\subsection*{Rips complex}
\begin{equation}\label{eq:rips}
	\mathrm{Rips_\epsilon}(X) = \{ S \subseteq X : S \neq \emptyset \;\mbox{ and }\;\mathrm{diam}(S) \leq \epsilon \}
\end{equation} 
Letting the scale parameter $\epsilon \in \mathbb{R}$ vary, one obtains a filtration of simplicial complexes connected by inclusion maps: 
$$ \mathrm{Rips_\epsilon}(X) \to \mathrm{Rips_{\epsilon'}}(X) \to \dots \to \mathrm{Rips_{\epsilon''}}(X)$$

%Given a Rips complex, 	$H_p(K_1) \to H_p(K_2) \to \dots \to H_p(K_m)$


\end{document}
