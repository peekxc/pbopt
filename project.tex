\documentclass[10pt]{article}
\usepackage[margin=0.5in]{geometry}
\usepackage{amsmath, amsthm, amssymb}
\usepackage{mathtools}
\usepackage{hyperref}
\usepackage{url}

\usepackage{titlesec}
\titlelabel{\thetitle.\quad}

\DeclareMathOperator*{\argmax}{arg\,max}
\DeclareMathOperator*{\argmin}{arg\,min}

\newtheorem{remark}{Remark}
%\let\oldref\ref
%\renewcommand{\ref}[1]{((\oldref{#1}))}

\newcommand\sbullet[1][.5]{\mathbin{\vcenter{\hbox{\scalebox{#1}{$\bullet$}}}}}
\newcommand\numberthis{\addtocounter{equation}{1}\tag{\theequation}}
\newcommand{\bigzero}{\mbox{\normalfont\Large\bfseries 0}}
\newcommand{\rvline}{\hspace*{-\arraycolsep}\vline\hspace*{-\arraycolsep}}

\title{\vspace{-2.0em} \vspace{-0.5em}}
\author{Matt Piekenbrock}
\date{}

\begin{document}
\noindent

\section{Motivation}
$<$ insert motivating examples, etc $>$

% Suppose one observes points in a geometric space whose position is driven by some unknown continuous-time system. 
% Towards understanding its dynamic, one may ask whether one can infer properties of the underlying evolving system
\section{Background \& Notation}
% \textbf{Persistent Homology:}
A \emph{simplicial complex} $K$ is a collection of simplices $\{\sigma\}$ such that $\tau \subseteq \sigma \in K \implies \tau \in K$. A $p$-chain is a formal $\mathbb{F}$-linear combination of $p$-simplices of $K$. The collection of $p$-chains under addition yields an $\mathbb{F}$-vector space denoted $C_p(K)$. 
The $p$-boundary $\partial_p(\sigma)$ of a $p$-simplex $\sigma\in K$ is the alternating sum of its oriented co-dimension 1 faces, and the $p$-boundary of a $p$-chain is defined linearly in terms of its constitutive simplices. 
A $p$-chain with zero boundary is called a $p$-cycle, and together they form $Z_p(K) = \mathrm{Ker}\,\partial_p$. Similarly, the collection of $p$-boundaries forms  $B_p(K) = \mathrm{Im}\,\partial_{p+1}$. Since $\partial_p \circ \partial_{p+1} = 0$ for all $p\geq 0$, then the quotient space $H_p(K) = Z_p(K) / B_{p}(K)$ is well-defined, and called the $p$-th homology of $K$ with coefficients in $\mathbb{F}$. 

A \emph{filtration} $K_\bullet$ of a simplicial complex $K$ is a function $f : K \to \mathbb{R}$ satisfying $f(\tau) \leq f(\sigma)$ whenever $\tau \subseteq \sigma$. Connecting a sequence of simplices $[\sigma_i]_{i=1, \dots, m}$ ordered increasingly by $f$ by inclusion yields such a family of complexes:
\begin{equation}
	\emptyset = K_0 \subsetneq K_1 \subsetneq \dots \subsetneq K_m  = K_\bullet
\end{equation} 
where $K_i  = K_{i-1} \cup \{\sigma_i\}$. For every pair $i,j \in [m]$ with $i < j$, the inclusions $K_i \subsetneq K_{i+1} \subsetneq \dots \subsetneq K_j$ induce linear transformations $h_p^{i,j}$  at the level of homology:
\begin{equation}
	0 = H_p(K_1) \to H_p(K_2) \to \dots \to H_p(K_i) \underbracket[0.5pt]{\to \dots \to}_{h_p^{i,j}} H_p(K_j) \to \dots \to H_p(K_m) = H_p(K_\bullet) 
\end{equation}
The $p$-th persistent homology groups are the images of these transformations: $H_{p}^{i,j} = \mathrm{Im}\,h_p^{i,j}$. 
When persistent homology is considered with field coefficients $\mathbb{F}$, the sequence of homology groups admits a unique decomposition of $K_\bullet$ into a pairing of simplices $(\sigma_i, \sigma_j)$~\cite{}. These pairs demarcate the evolution of homology classes: $\sigma_i$ marks the creation of a homology class, $\sigma_j$ marks its destruction, and the difference $\lvert i - j \rvert$ records the lifetime of the class, called its \emph{persistence}. For a fixed $p \geq 0$, the collection of persistent pairs $(f(\sigma_i), f(\sigma_j))$ together with unpaired simplices $(f(\sigma_l), \infty)$ form a summary representation $\mathrm{dgm}_p(K_\bullet)$ called the \emph{$p$-th persistence diagram of $K_\bullet$}.

\begin{remark}
\normalfont In practice, filtrations often arise from triangulations parameterized by geometric scaling parameters, and references to the ``persistence'' of a given homology class actually refers to its lifetime $\lvert f(\sigma_i) - f(\sigma_j) \rvert$ with respect to the scaling parameter.
$<$TODO: introduce Rips, switch notation accordingly $>$

For a fixed $t \in \mathrm{T}$, the \emph{Vietoris-Rips} complex at scale $\epsilon \in \mathbb{R}$ is the abstract simplicial complex given by 
$$\mathrm{Rips_{\epsilon}}(d_X(t)) := \{ \sigma \subset X : d_X(t)(x, x') \leq \epsilon \text{ for all } x, x' \in \sigma \} $$ 
\noindent Connecting successive complexes via inclusion maps $\mathrm{Rips_{\epsilon}}(d_X(t)) \hookrightarrow \mathrm{Rips_{\epsilon'}}(d_X(t))$ for $\epsilon < \epsilon'$ yields a family of complexes $\mathrm{Rips}_{\alpha} := \{ \, \mathrm{Rips}_\epsilon(d_X(t)) \, \}_{\epsilon \leq \alpha}$ which we call the \emph{Vietoris-Rips filtration at time $t$}. 
These inclusions induce linear maps at level of homology, i.e. 
$$ \mathrm{H}_p(\mathrm{Rips}_{\epsilon}(d_X(t))) \to \mathrm{H}_p(\mathrm{Rips}_{\epsilon'}(d_X(t))) \to \dots \to \mathrm{H}_p(\mathrm{Rips}_{\alpha}(d_X(t)))$$ 
where $0 \leq \epsilon \leq \epsilon' \leq \alpha$. 

\end{remark} 

%Note that if $i = j$, then $H_{p}^{i,j} = H_{p}(K_i) = H_{p}(K_i)$ is   just the ``standard'' homology. 
% Simplices whose inclusion in the filtration creates a new homology class are called \emph{creators}, and simplices that destroy homology classes are   called \emph{destroyers}. 
% The filtration indices of these creators/destroyers are referred to as \emph{birth} and \emph{death} times, respectively. 
%The collection of birth/death pairs $(i,j)$ is denoted $\mathrm{dgm}_p(K)$, and referred to as the $p$-th \emph{persistence diagram} of $K$.
%If a homology class is born at time $i$ and dies entering time $j$, the difference $\lvert i - j \rvert$ is called the \emph{persistence} of that class.



\section{Introduction}

 %by tracing changes in the geometry of 

% how is the topology and geometry of the of underlying space. 

% Towards understanding the behavior of such a collection of points 

Suppose one is given a time-varying metric space $(X, d_X(\cdot))$ and is interested in finding an instance in time where the persistence diagram contains many highly persistent pairs. 


Formally, let $\delta_X$ denote an $\mathrm{T}$-parameterized metric space $\delta_X = ( X, d_X(\cdot) )$, where $d_X: \mathrm{T} \times X \times X \to \mathbb{R}_+$ is called a \emph{time-varying metric}  and $X$ is a finite set with fixed cardinality $\lvert X \rvert = n$. $\delta_X$ as called a \emph{dynamic metric space} (DMS) iff $d_X(\cdot)(x, x'): \mathbb{R} \to \mathbb{R}_{+}$ is continuous for every pair $x, x' \in X$ and $(X, d_X(t))$ is a pseudo-metric space for every $t \in \mathrm{T}$. 


The $p$th \emph{Betti number} is defined as the dimension of any of these homology groups $\beta_p = \mathrm{dim}(H_p(\mathrm{Rips}_{\alpha})))$. 

By restricting our attention to the persistent homology groups which were born before $b \in \mathbb{R}$ and died after $d \in \mathbb{R}$, we obtain the $p$-th \emph{persistent Betti number} with respect to $(b,d)$ as a function of $t \in \mathrm{T}$: 
$$ \beta_{p}^{b,d}(t) = \left(\mathrm{dim} \circ \mathrm{H}_p^{i,j} \circ \mathrm{Rips} \circ d_X \right)(t)$$
This quantity can be readily visualized as the number of persistent pairs lying inside the box $[0, b] \times (d, \infty)$  on the collection of all persistence diagrams for varying $t \in \mathrm{T}$.
We consider the problem of maximizing the $p$-th \emph{persistent} Betti number $\beta^{b,d}_p$ over $\mathrm{T}$: 
\begin{equation}
	t_\ast = \argmax_{t \in \mathrm{T}}	 \beta_{p}^{b,d}(t)
\end{equation}
Since Betti numbers are integer-valued invariants, direct optimization is difficult. Moreover, the space of persistence diagrams is [banach space statement]....
Nonetheless, the differentiability of persistence has been studied extensively in [show chain rule paper on persistence diagrams]...

\subsection*{A motivating derivation}
For the moment, we omit the subscript $t \in \mathrm{T}$ and focus on a fixed instance of time. Let $B_p(K_\ast) \subseteq Z_p(K_\ast) \subseteq C_p(K_\ast)$ denote the $p$-th boundary, cycle, and chain groups of $K_\ast$, respectively. 
Given a simplicial filtration $K_{\bullet}$, let boundary operator $\partial_p : C_p( K_{\bullet}) \to C_p(K_{\bullet})$ denote the boundary operator sending $p$-chains to their respective boundaries. 
With a slight abuse of notation, we use $\partial_p$ to also denote the filtration boundary matrix with respect to the ordered basis $(\sigma_i)_{1 \leq i \leq m_p}$.  
The $p$-th persistent Betti number between scales $(b,d)$ is defined as: 
\begin{align*}
	\beta_p^{b,d} &= \mathrm{dim}(H_p^{b,d}) \\
	&= \mathrm{dim} \left( Z_p(K_b) / (Z_p(K_b) \cap B_p(K_d) \right) \\
	& \numberthis = \mathrm{dim} \left( Z_p(K_b) \right) - \mathrm{dim}\left( Z_p(K_b) \cap B_p(K_d) \right ) \label{eq:pb2}
\end{align*}
Note we may rewrite~\eqref{eq:pb2} with a straightforward application of the rank-nullity theorem:
\begin{equation} \label{eq:pb3}
\beta_p^{b,d} = \mathrm{dim} \left( C_p(K_b) \right) - \mathrm{dim} \left( B_p(K_b) \right) - \mathrm{dim}\left( Z_p(K_b) \cap B_p(K_d) \right )  
\end{equation}
Let $\partial_p^{b}$ and $\partial_p^{b, d}$ denote matrices whose columns span the subspaces $B_p(K_b)$ and $Z_p(K_b) \cap B_p(K_d)$, respectively. We address their computation in section~\eqref{}. Observe that equation~\eqref{eq:pb3} can be written as: 
\begin{align}
	\beta_p^{b,d} &= \lvert \, \partial_p^b \, \rvert - \mathrm{rank}(\partial_p^b) - \mathrm{rank}(\partial_p^{b,d}) \\
	&= \lvert \, \partial_p^b \, \rvert - \left( \mathrm{rank}(\partial_p^b) + \mathrm{rank}(\partial_p^{b,d}) \right) \\
	&=
	\lvert \, \partial_p^b \, \rvert - 
	\arraycolsep=1.8pt\def\arraystretch{1.25}
	\mathrm{rank}\left(\left[\begin{array}{c|c}
 		\partial_p^{b} & 0 \\
		\hline
		0 & \partial_p^{b,d}
	\end{array}\right] \right) \label{eq:block_pb}
\end{align}
where here we use $\lvert \, M \, \rvert = \mathrm{dim}(\mathrm{dom}(M))$. Thus, $\beta_p^{b,d}$ is expressible as a difference between a simple-to-compute quantity ($\lvert \, \partial_p^b \, \rvert$ simply counts the number of $p$-simplices satisfying $\{ \mathrm{diam}(\sigma) \leq b \}$ for some fixed $b \in \mathbb{R}_+$) and the rank of a particular block matrix. We make use of this fact in what following section. 
\\
\\
% Thus, after suitable normalization, the right-most term of equation~\ref{eq:block_pb} can be relaxed to a convex function. 
% The columns of $\partial_p^\ast$ span $C_p(X_\ast)$, thus 
% Let $\mathcal{N}(\cdot)$ and $\mathcal{R}(\cdot)$ denote the null-space of column-space of its arguments, respectively. 
\noindent \textbf{Continuous Relaxation:} 
The $p$-th persistent Betti number is informative in capturing the necessary conditions of our goal: ...

However, as an integer-valued invariant, Betti numbers pose several difficulties to direct optimization. 

Ideally, we would like an expression akin to equation~\eqref{eq:block_pb} that is amenable to optimization.

A remarkable result established by~\cite{} show that the $\mathrm{rank}(\cdot)$ function is lower-bounded by the convex envelope... [describe this more in detail]

\section{Computation}
\subsection*{Bases computation}
In this section, we discuss the computation of suitable bases for the subspaces $Z_p(X_\ast)$, $B_p(X_\ast)$, and $Z_p(X_\ast) \cap B_p(X_\ast)$. In particular, we address two cases: the \emph{dense} case, wherein the aforementioned bases are represented densely in memory, and the \emph{sparse} case, which uses the structure of a particular decomposition of the boundary matrices to derive bases whose size in memory inherits the sparsity pattern of the decomposition.
\\
\\
\textbf{Dense case:} 
$<$ TODO $>$

\textbf{Sparse case:} 
$<$ TODO $>$

\appendix
\section{Appendix}

\subsection*{Dynamic Metric Spaces}
Consider an $\mathbb{R}$-parameterized metric space $\delta_X = ( X, d_X(\cdot) )$ where
$X$ is a finite set and $d_X(\cdot): \mathbb{R} \times X \times X \to \mathbb{R}_{+}$, satisfying: 
\begin{enumerate}
	\item For every $t \in \mathbb{R}, \delta_X(t) = (X, d_X(t))$ is a pseudo-metric space\footnote{This is required so that if one can distinguish the two distinct points $x, x' \in X$ incase $d_X(t)(x, x') = 0$ at some $t \in \mathbb{R}$. } 
	\item For fixed $x, x' \in X$, $d_X(\cdot)(x, x'): \mathbb{R} \to \mathbb{R}_{+}$ is continuous.
\end{enumerate}
When the parameter $t \in \mathbb{R}$ is interpreted as \emph{time}, the above yields a natural characterization of a ``time-varying'' metric space. More generally, we refer to an $\mathbb{R}^h$-parameterized metric space as \emph{dynamic metric space}(DMS). Such space have been studied more in-depth~\cite{} and have been shown...
 
\subsection*{Homology}
Let $K$ be an abstract simplicial complex and $\mathbb{F}$ a field. A $p$-chain is a formal $\mathbb{F}$-linear combination  of $p$-simplices of $K$. The collection of $p$-chains under addition yields an $\mathbb{F}$-vector space denoted $C_p(K)$. 
The $p$-boundary $\partial_p(\sigma)$ of a $p$-simplex $\sigma\in K$ is the alternating sum of its oriented co-dimension 1 faces, and the $p$-boundary of a $p$-chain is defined linearly in terms of its constitutive simplices. 
A $p$-chain with zero boundary is called a $p$-cycle, and together they form $Z_p(K) = \mathrm{Ker}\,\partial_p$. Similarly, the collection of $p$-boundaries forms  $B_p(K) = \mathrm{Im}\,\partial_{p+1}$. Since $\partial_p \circ \partial_{p+1} = 0$ for all $p\geq 0$, then the quotient space $H_p(K) = Z_p(K) / B_{p}(K)$ is well-defined, and called the $p$-th homology of $K$ with coefficients in $\mathbb{F}$. 
If $\{K_i\}_{i\in [m]}$ is a filtration, then the inclusion maps  $K_i\subset K_{i+1}$ induce linear transformations 
%$f_p^{i,j}: H_p(K_i) \to H_p(K_j)$ 
at the level of homology:
\begin{equation}
	H_p(K_1) \to H_p(K_2) \to \dots \to H_p(K_m)
\end{equation}
%The $p$-th persistent homology groups are the images of these transformations: $H_{p}^{i,j} = \mathrm{Im}\,f_p^{i,j}$. 
%Note that if $i = j$, then $H_{p}^{i,j} = H_{p}(K_i) = H_{p}(K_i)$ is   just the ``standard'' homology. 
Simplices whose inclusion in the filtration creates a new homology class are   called \emph{creators}, and simplices that destroy homology classes are   called \emph{destroyers}. 
The filtration indices of 
these creators/destroyers are referred to as \emph{birth} and \emph{death} times, respectively. 
The collection of birth/death  pairs 
$(i,j)$ is denoted $\mathrm{dgm}_p(K)$, 
and referred to as the $p$-th \emph{persistence diagram} of $K$.
If a homology class is born at time $i$ and dies entering time $j$, the difference $\lvert i - j \rvert$ is called the \emph{persistence} of that class.
In practice, filtrations often arise from triangulations parameterized by geometric scaling parameters, and the ``persistence'' of a homology class actually refers to its lifetime with respect to the scaling parameter. 


\subsection*{Rips complex}
\begin{equation}\label{eq:rips}
	\mathrm{Rips_\epsilon}(X) = \{ S \subseteq X : S \neq \emptyset \;\mbox{ and }\;\mathrm{diam}(S) \leq \epsilon \}
\end{equation} 
Letting the scale parameter $\epsilon \in \mathbb{R}$ vary, one obtains a filtration of simplicial complexes connected by inclusion maps: 
$$ \mathrm{Rips_\epsilon}(X) \to \mathrm{Rips_{\epsilon'}}(X) \to \dots \to \mathrm{Rips_{\epsilon''}}(X)$$

%Given a Rips complex, 	$H_p(K_1) \to H_p(K_2) \to \dots \to H_p(K_m)$


\end{document}
